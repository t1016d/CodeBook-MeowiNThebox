%\documentclass[10pt,twocolumn,oneside]{article}
\documentclass[12pt,twocolumn,oneside]{article}
\setlength{\columnsep}{20pt}                                                                    %兩欄模式的間距
\setlength{\columnseprule}{0pt}                                                                %兩欄模式間格線粗細

\usepackage{amsthm}								%定義,例題
\usepackage{amssymb}
%\usepackage[margin=2cm]{geometry}
\usepackage{fontspec}								%設定字體
\usepackage{color}
\usepackage[x11names]{xcolor}
\usepackage{xeCJK}								%xeCJK
\usepackage{listings}								%顯示code用的
%\usepackage[Glenn]{fncychap}						%排版,頁面模板
\usepackage{fancyhdr}								%設定頁首頁尾
\usepackage{graphicx}								%Graphic
\usepackage{enumerate}
\usepackage{titlesec}
\usepackage{amsmath}

%\usepackage[T1]{fontenc}
\usepackage{amsmath, courier, listings, fancyhdr, graphicx}
\topmargin=0pt
\headsep=5pt
\textheight=780pt
\footskip=0pt
\voffset=-40pt
\textwidth=545pt
\marginparsep=0pt
\marginparwidth=0pt
\marginparpush=0pt
\oddsidemargin=0pt
\evensidemargin=0pt
\hoffset=-42pt

%\renewcommand\listfigurename{圖目錄}
%\renewcommand\listtablename{表目錄} 

%%%%%%%%%%%%%%%%%%%%%%%%%%%%%

\setmainfont{Consolas}				%主要字型
\setCJKmainfont{Consolas}			%中文字型
%\setmainfont{sourcecodepro}
\XeTeXlinebreaklocale "zh"						%中文自動換行
\XeTeXlinebreakskip = 0pt plus 1pt				%設定段落之間的距離
\setcounter{secnumdepth}{3}						%目錄顯示第三層

%%%%%%%%%%%%%%%%%%%%%%%%%%%%%
\makeatletter
\lst@CCPutMacro\lst@ProcessOther {"2D}{\lst@ttfamily{-{}}{-{}}}
\@empty\z@\@empty
\makeatother
\lstset{											% Code顯示
language=C++,										% the language of the code
basicstyle=\footnotesize, 						% the size of the fonts that are used for the code
numbers=left,										% where to put the line-numbers
numberstyle=\footnotesize,						% the size of the fonts that are used for the line-numbers
stepnumber=1,										% the step between two line-numbers. If it's 1, each line  will be numbered
numbersep=5pt,										% how far the line-numbers are from the code
backgroundcolor=\color{white},					% choose the background color. You must add \usepackage{color}
showspaces=false,									% show spaces adding particular underscores
showstringspaces=false,							% underline spaces within strings
showtabs=false,									% show tabs within strings adding particular underscores
frame=false,											% adds a frame around the code
tabsize=2,											% sets default tabsize to 2 spaces
captionpos=b,										% sets the caption-position to bottom
breaklines=true,									% sets automatic line breaking
breakatwhitespace=false,							% sets if automatic breaks should only happen at whitespace
escapeinside={\%*}{*)},							% if you want to add a comment within your code
morekeywords={*},									% if you want to add more keywords to the set
keywordstyle=\bfseries\color{Blue1},
commentstyle=\itshape\color{Red4},
stringstyle=\itshape\color{Green4},
}

%%%%%%%%%%%%%%%%%%%%%%%%%%%%%

\begin{document}
\pagestyle{fancy}
\fancyfoot{}
%\fancyfoot[R]{\includegraphics[width=20pt]{ironwood.jpg}}
\fancyhead[L]{National Taiwan University - MeowiNThebox}
\fancyhead[R]{\thepage}
\renewcommand{\headrulewidth}{0.4pt}
\renewcommand{\contentsname}{Contents} 

\scriptsize
\tableofcontents
%%%%%%%%%%%%%%%%%%%%%%%%%%%%%

\newpage

\section{Basic}
\subsection{default code}
\lstinputlisting{../default.cpp}

\subsection{.vimrc}
\lstinputlisting{../.vimrc}

\section{math}
\subsection{ext gcd}
\lstinputlisting{../math/ext_gcd.cpp}

\subsection{FFT}
\lstinputlisting{../math/FFT.cpp}

\subsection{MillerRabin other}
\lstinputlisting{../math/MillerRabin_other.cpp}

\subsection{Guass}
\lstinputlisting{../math/guass.cpp}

\section{flow}
\subsection{dinic}
\lstinputlisting{../flow/dinic.cpp}

\section{string}
\subsection{KMP}
\lstinputlisting{../string/KMP.cpp}

\subsection{Z-value}
\lstinputlisting{../string/Z-value.cpp}

\subsection{Z-value-palindrome}
\lstinputlisting{../string/Z-value-palindrome.cpp}

\subsection{Suffix Array(\(O(N log N)\))}
\lstinputlisting{../string/Suffix_Array(NlogN).cpp}

\subsection{Aho-Corasick}
\lstinputlisting{../string/Aho-Corasick.cpp}

\subsection{Aho-Corasick-2016ioicamp}
\lstinputlisting{../string/Aho-Corasick-2016ioicamp.cpp}

\section{graph}
\subsection{Bipartite matching(\(O(N^3)\))}
\lstinputlisting{../graph/Bipartite_matching(N^3).cpp}

\subsection{KM(\(O(N^4)\))}
\lstinputlisting{../graph/KM.cpp}

\subsection{Max clique(bcw)}
\lstinputlisting{../graph/Maximum_Clique_bcw.cpp}

\subsection{EdgeBCC}
\lstinputlisting{../graph/edgeBCC.cpp}

\subsection{VerticeBCC}
\lstinputlisting{../graph/verticeBCC.cpp}

\subsection{Them.}
\lstinputlisting{../graph/them.tex}

\section{data structure}
\subsection{Treap}
\lstinputlisting{../datastructure/treap.cpp}

\subsection{copy on write treap}
\lstinputlisting{../datastructure/copy_on_write_treap.cpp}

\subsection{copy on write segment tree}
\lstinputlisting{../datastructure/copy_on_write_segment_tree.cpp}

\subsection{Treap+(HOJ 92)}
\lstinputlisting{../datastructure/HOJ_92(treap).cpp}

\subsection{Leftist Tree}
\lstinputlisting{../datastructure/leftist_tree.cpp}

\subsection{Link Cut Tree}
\lstinputlisting{../datastructure/LCT.cpp}

\subsection{Heavy Light Decomposition}
\lstinputlisting{../datastructure/ShuLian.cpp}

\subsection{Disjoint Sets + offline skill}
\lstinputlisting{../datastructure/DJS.cpp}

\section{geometry}
\subsection{Basic}
\lstinputlisting{../geometry/geometry.cpp}

\subsection{Smallist circle problem}
\lstinputlisting{../geometry/SmallistCircleProblem.cpp}

\section{Others}
\subsection{Random}
\lstinputlisting{../others/random.cpp}

\subsection{Fraction}
\lstinputlisting{../others/frac.cpp}

\end{document}
